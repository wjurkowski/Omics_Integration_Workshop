\documentclass[11pt, oneside]{article}   	% use "amsart" instead of "article" for AMSLaTeX format
\usepackage{geometry}                		% See geometry.pdf to learn the layout options. There are lots.
\geometry{a4paper}                   		% ... or a4paper or a5paper or ...letterpaper 
%\geometry{landscape}                		% Activate for rotated page geometry
%\usepackage[parfill]{parskip}    		% Activate to begin paragraphs with an empty line rather than an indent
\usepackage{graphicx}				% Use pdf, png, jpg, or eps§ with pdflatex; use eps in DVI mode
								% TeX will automatically convert eps --> pdf in pdflatex		
\usepackage{amssymb}
\DeclareGraphicsExtensions{.pdf,.png,.jpg}
\graphicspath{ {images/} }

\usepackage{hyperref}
\usepackage{subcaption}
\usepackage{amsmath}
\usepackage{framed}
\usepackage{tabulary}
\usepackage{hyperref}
\usepackage{outlines}

\newcommand{\norm}[1]{\left\lVert#1\right\rVert}


\title{Connecting Nutrition, Health and Environment }
\author{Wiktor Jurkowski, Perla Troncoso Rey, Earlham Institute}
\date{25 - 26 January 2017}	 % Activate to display a given date or no date

\begin{document}
\maketitle

\tableofcontents

\listoffigures
\listoftables


\section{Introduction}

In the first part of this practical session we will see general techniques to explore the patterns or structure of the data using Principal Component Analysis (PCA), and Hierarchical Clustering. We will then look into compiling an interaction network with PSICQUIC \cite{Aranda2011} and how to visualise the network using Cytoscape \cite{Shannon2003} \cite{Smoot2011}.

In the second part, we will look at ways to rank genes (and/or metabolites) using approaches based on logistic regression. 

We will use a publicly available data from a study on fatty liver disease of obese and lean human subjects \cite{Wruck2015}.



\part{Exploring the expression data}

One could obtain gene expression from microarrays or RNA sequencing data. It is not within the scope of this session to look at the details of obtaining quantifying the expression of genes from microarrays or RNA sequencing data but instead we will start our analysis assuming that expression data has been quantified.
The gene expression data is normally stored in tabular file, representing a matrix where the columns are the samples or experiments, and the rows represent the genes. 

Example of expression data is shown in \autoref{fig:ExpressionData}. The table shows the expression of six genes in four different experiments or samples. This is, gene A has expression of 0.1 for sample 1, 0.8 for sample 2, 0.3 for sample 3, and so forth. 

\begin{figure}[!ht]
	\centering
	\includegraphics[width=0.7\textwidth]{example_expression_data}
	\caption{Example of expression data with samples across the columns and individual genes down the rows.}
	\label{fig:ExpressionData}
\end{figure}


Expression data can be obtained using different algorithms. One of the most well know are TopHat and Cufflinks protocol for the analysis of RNA sequencing data, which includes quantification of gene expression. \autoref{tab:ExpressionCufflinks} shows an example of the output provided by cufflinks with the estimated gene-level expression values. Cufflinks uses the notation ``XLOC\_{\it numeric\_sequence}'' to identify a gene.


\begin{table}[ht]
    \centering
    \caption{Example of the output provided by cufflinks for the quantification of gene expression from RNA sequencing data.}
    \begin{tabular}{c|c|c|c|c|c|c|c|c}
    tracking\_id & sample1 & sample2 & sample3 & sample4 & sample5 & sample6 & sample7 & sample8 \\
    \hline
    XLOC\_000001 & 35.1077 & 50.9662 & 78.7724 & 35.4736 & 69.6067 & 63.9241 & 57.7967 & 61.4227 \\
    XLOC\_000002 & 49.7359 & 64.6178 & 46.8884 & 74.617 & 66.0371 & 42.9654 & 645.65 & 64.8351 \\
    XLOC\_000003 & 0 & 0 & 0.937767 & 0 & 0 & 0 & 0 & 0\\
    XLOC\_000004 & 89.7196 & 85.5504 & 185.678 & 74.617 & 142.783 & 168.718 & 172.63 & 167.206 \\
    XLOC\_000005 & 12.6778 & 39.1347 & 158.483 & 22.0181 & 28.5566 & 45.0613 & 15.9701 & 50.0481 \\
    XLOC\_000006 & 10.7273 & 9.1011 & 10.3154 & 13.4555 & 7.13915 & 6.28762 & 7.60483 & 12.512 \\
    XLOC\_000007 & 0 & 0 & 0.937767 & 0 & 0 & 0 & 0 & 0 \\
    XLOC\_000008 & 55.5871 & 37.3145 & 86.2746 & 66.0544 & 66.9295 & 53.4448 & 54.7548 & 75.0722 \\
    XLOC\_000009 & 37.0581 & 16.382 & 24.3819 & 24.4646 & 38.3729 & 15.7191 & 24.3355 & 50.0481 \\
    XLOC\_000010 & 812.352 & 483.269 & 696.761 & 748.616 & 1094.97 & 521.873 & 675.309 & 741.622 \\
    XLOC\_000011 & 0 & 0 & 0 & 0 & 0 & 1.04657 & 0.760483 & 1.13746\\
    \hline
    \label{tab:ExpressionCufflinks}
    \end{tabular}
\end{table}




\section{Principal Component Analysis}

Principal Component Analysis, commonly known as PCA, is a mathematical technique that is used to explore data, specially high-dimensional data, to extract the most important trends in the data.

When thinking of gene expression data, high dimensionality comes from the large number of dimensions of the data. This is, the result of each experiment can be thought as a kind of space, where each each feature is a coordinate in the space. There are typically thousands of genes (dimensions) and the structure of pattern in the data extends to all the dimensions. 


\paragraph{How PCA works}
\paragraph{}

The mean represents the average of the values in the data:

\begin{equation}
   \bar{{\bf X}} = \frac{1}{n} \sum_{i=1}^{n} x_i 
\end{equation}

The variance provides the the spread of the data:

\begin{equation}
   \text{Var} ({\bf X}) = \sigma^2  = \frac{1}{n-1} \sum_{i=1}^{n} ( x_i -  \bar{{\bf X}})^2
\end{equation}


For example, Figure \ref{fig:MeanVariance} shows two distributions with the same mean but different variance. This means that the data points are at the same location but with a different strength. Thus, the third statistic we'll need is the covariance.

\begin{figure}[!ht]
	\centering
	\includegraphics[width=0.7\textwidth]{same-mean_different-variance}
	\caption{Two distributions with the same mean but different variance.}
	\label{fig:MeanVariance}
\end{figure}


The covariance represents the degree of co-dependence of two variables, i.e., it measures the co-dependency of two variables, given by:

\begin{equation}
   \text{Cov} ({\bf X}, {\bf Y}) = \frac{1}{n-1} \sum_{i=1}^{n} (x_i - \bar{ {\bf X}}) (y_i - \bar{ {\bf Y}})
\end{equation}

Increases with increasing co-dependency and variance. Just as the variance measures the degree to which a set of data varies, the co-variance is a measure of the way two sets of data vary together.

\begin{equation}
   \text{Cov} ({\bf X}, {\bf X}) = \text{Var}({\bf X})
\end{equation}

The covariance also increases in magnitude as the variance of each of the two datasets increases.
Correlation values can be negative or positive, indicating whether the values of two variables increase or decrease together. Figure \ref{fig:examples-correlation} shows some examples of correlation.

\begin{figure}[!ht]
	\centering
	\includegraphics[width=0.7\textwidth]{examples-correlation}
	\caption{Examples of correlation}
	\label{fig:examples-correlation}
\end{figure}



\paragraph{Coordinate transformations}
\paragraph{}

In a two dimensional space described by coordinates, a point in space is described by X and Y such that ${\bf v} = [x_1, y_1]$. For example, the vector $v_1 = [1 ~2]^T$ represents a point in the 2 dimensional space as shown in Figure \ref{fig:CoordinateTransform} (a).
An alternative coordinate system described by the coordinates $\text{X}^\prime$ and $\text{Y}^\prime$, has a different column vector describing the same point ${\bf v^\prime} = [x_1^\prime, y_1^\prime]$, shown in Figure \ref{fig:CoordinateTransform} (b).

The two coordinate systems are $T{\bf v} = {\bf v}^\prime$, related to the orthogonal transform matrix $T$. An orthogonal matrix is the kind of matrix which performs rotated-axis coordinate transforms. Thus, we can make a new coordinate system by using a transformation matrix T, which relates the two coordinates vectors by matrix multiplication. There are many types of transformations but we are particularly interested in transformations which rotate the coordinate axis. These are performed by matrices which have the property called orthogonality.

\begin{figure}[!ht]
	\centering
	\includegraphics[width=0.7\textwidth]{example-coordinate-transform}
	\caption{Example of a coordinate transform}
	\label{fig:CoordinateTransform}
\end{figure}



\paragraph{Eigenvalues and Eigenvectors}
\paragraph{}

When a transformation matrix maps a vector to a multiple of itself, then the vector is called an Eigenvector. The amount by which the vector is multiplied (stretched) is the associated Eigenvalue:

\begin{equation}
T x = \lambda x
\end{equation} 

\noindent where $\lambda$ are the Eigenvalues and $x$ are the Eigenvectors. 

In general terms, PCA uses covariance to encode the structure in the data and then eigenvectors to devise a new set of coordinates that best reveals the structure by finding the appropriate set of directions. One result from linear algebra is that if the eigenvectors are placed next to each other to construct an orthogonal matrix that performs a coordinate transformation. It is important to mention that Eigenvectors are placed in descending order of their corresponding eigenvalue to ensure that the first components encode most of the variance in the data.
The transpose of this matrix of Eigenvectos is an orthogonal matrix which performs a rotated-axis coordinate transformation. We can transform our data matrix, $D$, to the new coordinates, $D_{PCA}$:  

\[
   D_{PCA} = W^T D
\]


For example: 

\[ \text{The matrix: }
%
   \left(
      \begin{tabular}{cc}
      1 & 3 \\ 
      2 & 2
      \end{tabular}
   \right)
   %
   \text{has eigenvalues 4 and -1}
   \text{ and the eigenvectors}
   \left( \begin{array}{c}
      1 \\ 
      1
      \end{array}
   \right)
   \text{and} 
   \left( \begin{array}{c}
      3 \\ 
      -2
      \end{array}
   \right)
\]


such that

\[ \left( \begin{array}{cc}
      1 & 3\\ 
      2 & 2
      \end{array} \right)
%
   \left( \begin{array}{c}
      1 \\ 
      1
      \end{array} \right)
%
   = 4
   \left( \begin{array}{c}
   1\\
   1
   \end{array} \right)
\text{~~~and~~~}
   \left( \begin{array}{cc}
   1 & 3\\
   2 & 2
   \end{array} \right)
%
   \left( \begin{array}{c}
   3\\
   -2
   \end{array} \right)
%
   =-1
   \left( \begin{array}{c}
   3\\
   -2
   \end{array} \right)
\]

The orthogonal matrix using these Eigenvector is: 

\[ 
  W^T = 
   \left[ \begin{array}{cc}
      1 & 1 \\ 
      3 & -2      
   \end{array} \right]
\]



% Here we start with some hands-on exercises

% Example 1
\subsection{Exercise: PCA for the expression of two genes}
\paragraph{}

We will use R and the package stats to perform PCA. We will use an example data which represents several measurements of the expression of two genes, $x$ and $y$, with the following values:

\begin{center}
\begin{tabular}{c|c}
   x & y \\
   \hline
   2.5 & 2.4\\
   0.5 & 0.7\\
   2.2 & 2.9\\
   1.9 & 2.2\\
   3.1 & 3.0\\
   2.3 & 2.7\\
   2.0 & 1.6\\
   1.0 & 1.1\\
   1.5 & 1.6\\
   1.1 & 0.9\\
   \hline
\end{tabular}
\end{center}
   

We start by create a matrix of points in 2-d space (gene expression data) by using the following syntax:

%% create a matrix with the data from two genes
% script
% x <- c(2.5, 0.5, 2.2, 1.9, 3.1, 2.3, 2.0, 1.0, 1.5, 1.1)
% y <- c(2.4, 0.7, 2.9, 2.2, 3.0, 2.7, 1.6, 1.1, 1.6, 0.9)
% class <- c('control', 'control', 'control', 'control', 'control',
%  'case', 'case', 'case', 'case', 'case')
% ExpData <- data.frame(x = x, y = y, class = class)
	
\begin{framed}
\begin{verbatim}
   #n number of samples
   name_gene_1 <- c(values_in_sample1, ..., _value_in_sampleN) 
   #m number of genes
   name_gene_2 <- c(gene_1, gene_2, ..., gene_m)    
   samples <- c(sample1, ..., sampleN)
   # expression matrix
   Exp <- data.frame(gene1 = name_gene1, ..., geneM = name_geneM)	
\end{verbatim}
\end{framed}


Then, we plot these two genes (see Figure \ref{fig:PlotTwoGenes}) using the command: 

\begin{framed}
\begin{verbatim}
   plot(x, y)
\end{verbatim}
\end{framed}


\noindent Trends are already apparent because data is simple but this is not usually the case. We then perform an statistical analysis using Principal Component Analysis. Before starting with PCA, it is best to first have centered the data with mean zero. This is, calculate the mean of each of the two variables and substracted to obtain centered data (shown in Figure \ref{fig:PlotTwoGenesCentered}). 


%Plot data and data centered
\begin{figure}[!ht]
	\centering
	\begin{subfigure}{.45\textwidth}
		\includegraphics[width=\textwidth]{example1-plot-two-genes}
		\caption{}
		\label{fig:PlotTwoGenes}
	\end{subfigure}
	\begin{subfigure}{0.45\textwidth}
		\includegraphics[width=\textwidth]{example1-plot-two-genes-centered}
		\caption{}
		\label{fig:PlotTwoGenesCentered}
	\end{subfigure}
	\begin{subfigure}{0.45\textwidth}
		\includegraphics[width=\textwidth]{example1_plot_2PC}
		\caption{}
		\label{fig:PlotTwoGenesNewCoordiantes}
	\end{subfigure}
	\caption{Plot: in (a) shows the expression of two genes, (b) the expression of the same two genes after centering the data (expression has zero mean),  (c) gene expression is plot in the new coordiantes (PCA)}
	\label{fig:PlotData}
\end{figure}


Now, let us calculate the covariance matrix. Covariance matrix for two variables:

\[ \left[ \begin{array}{cc}
      $Cov(x,x)$ & $Cov(x,y)$\\ 
      $Cov(y,x)$ & $Cov(y,y)$
      \end{array} \right]
\]

\noindent and Covariance matrix for our data:
\[
   \left[ \begin{array}{cc}
      0.016 & 0.615 \\ 
      0.615 & 0.716
      \end{array} \right]
\]

The Eigenvalues of this matrix are:  1.284 and 0.0490. Eigenvalues gives the relative variance of the data in the direction defined by the Eigenvectors. From the values we can inferred that most variation is in one direction. To calculate the Eigenvalues in R use type:

\begin{framed}
\begin{verbatim}
	eigen(covariance_matrix)
\end{verbatim}
\end{framed}

The corresponding eigenvector are then placed in a matrix in descending order of eigenvalue:

\[
   \left[ \begin{array}{cc}
	0.6778734 & $-0.7351787$ \\
	0.7351787 & 0.6778734
      \end{array} \right]
\]

The transpose of this Eigenvectos will perform the coordinate transformation:

\[
   W^T = 
   \left[ \begin{array}{cc}
	0.6778734 & 0.7351787 \\
	$-0.7351787$ & 0.6778734
      \end{array} \right]
\]

\noindent This is an orthogonal matrix which performs a rotated-axis coordinate transformation.
We can transform our data matrix so that the data is represented in the new coordinates:  

\[
   D_{PCA} = W^T D
\]

\noindent which in our example is:

\[
   D_{PCA} = 
   \left[ \begin{array}{cc}
	0.6778734 & 0.7351787 \\
	$-0.7351787$ & 0.6778734
   \end{array} \right]
   %
   \left[ \begin{array}{cccccccccc}
      0.69 & $-1.31$ & 0.39 & 0.09 & 1.29 & 0.49 & 0.19 & $-0.81$ & $-0.31$ & $-0.71$ \\
      0.49 & $-1.21$ & 0.99 & 0.29 & 1.09 & 0.79 & $-0.31$ & $-0.81$ & $-0.31$ & $-1.01$
   \end{array} \right]
\]

\[
   =
   %
   \left[ \begin{array}{cccccccccc}
      0.83 & $-1.8$ & 0.99 & 0.27 & 1.8 & 0.91 & $-0.099$ & $-1.1$ & $-0.44$ & $-1.2$ \\
      $-0.18$ & 0.14 & 0.38 & 0.13 & $-0.21$ & 0.18 & $-0.35$ & 0.046 & 0.018 & $-0.16$
   \end{array} \right]
\]


The we can plot our data in the new coordinates, as shown in Figure \ref{fig:PlotTwoGenesNewCoordiantes}, where each coordinate is called principal component. 
The first coordinate aligns with the direction in the expression space where has the most variation. Subsequent coordinates would align with directions with descending degrees of variation. This is why we are careful to order according to the size of the eigenvalues.
Thus, PCA is capturing as much variation in the first component as possible, then the same for the second coordinate, and so on.
In the case of our data, all the meaningful variation sees to have been captured with the first coordinate, or the first principal component. Specially compared to the second component which would seem to be random scatter. So we have reduced the dimensionality of our data from two to one. In cases when dealing with thousands of genes, PCA might be able to capture most of the variation of the data in with only two or three principal components. Thus making it easier to visualise it, which is one of the main motivations of performing PCA.



% Example 2: loading example data from file
\subsection{Exercise: PCA using example data}


In this example we will use a publicly available dataset to explore expression data. We will use the stats package in R for computing PCA and will show how to visualise it using the function ggbiplot, which was implemented by Vince Q Vu \cite{Vu2016}. 


Gene expression data is usually stored in a tab delimited text file. The extension of such files could be .csv, .soft, .xls(x), etc. Use Excel, R or MATLAB to open and preview the file. It is important to mention that gene expression values must be normalised before PCA plotting.

The dataset used in this example is from a study on non-alcoholic fatty liver disease, published in 2015 by Wruck et al. \cite{Wruck2015}. The transcriptomics data was extracted from nine from patients that were recruited in the Multidisciplinary Obesity Research project at the Medical University of Graz, Austria, or at the Interdisciplinary Adipositas Center at the Kantonsspital St Gallen, Switzerland. Each sample is described in table \ref{tab:ExpressionDataWruck}) in terms of gender, age, BMI, percent of steatosis and steatosis grouping.

\begin{table}[h]
	\centering
	\caption{Details of the samples for microarray data for a study in fatty liver \cite{Wruck2015}}
	\begin{tabular}{c | c | c | c | c | c}
		\hline
		ID & gender & age & BMI & \% steatosis & steatosis grouping \\
		\hline
                H0004 & f & 54 & 47 & 10 & obese, low steatosis \\
                H0007 & f & 33 & 51 & 40 & obese, high steatosis \\
                H0008 & m & 61 & 46 & 40 & obese, high steatosis \\
                H0009 & f & 48 & 49 & 5 - 10 & obese, low steatosis \\
                H0011 & f & 58 & 45 & 70 & obese, high steatosis \\
                H0012 & f & 50 & 35 & 0 & obese, low steatosis \\
                H0018 & f & 35 & 41 & 30 - 40 & obese, high steatosis \\
                H0021 & m & 49 & 41 & 0 & no steatosis \\
                H0022 & m & 45 & 49 & 40 & obese, high steatosis
		\label{tab:ExpressionDataWruck}
	\end{tabular}
\end{table}


The gene expression matrix, gene annotation and sample annotation can be found in the following files:

\begin{itemize}
   \item {\bf gene-expression-table.txt}: gene expression table. This table is available as reference but for the simplicity we will use the following files which contain the expression data separately from the gene annotation, the names of samples, and the steatosis groups per sample.
   \item {\bf gene-expression.txt}: numerical matrix for the gene expression values, where the columns represent genes and the rows represent the samples. This is the transpose of the expression matrix because the function requires the rows of the input matrix to be observations and the columns features, which means rows to be the gene expression profiles (samples) and columns to be the genes.
   \item {\bf gene-annotation.txt}: string vector containing the gene names for the expression data
   \item {\bf samples.txt}: name of each sample
   \item {\bf groups.txt}: name of the steatosis group for each sample
\end{itemize}


Now, we can load the data into R to begin the analysis. We can either load the gene expression table by typing:

% Load data from files
\begin{framed}
\begin{verbatim}
#Expression data is saved in a tabular txt file
#The data is in a numerical matrix with no headers
	ExpData <- read.delim(FileName, header = FALSE, sep = '\t', 
		stringsAsFactors = FALSE)
	Annotation <- read.delim(FileName, header = TRUE, sep = '\t', 
		stringsAsFactors = FALSE)
	samples <- read.delim(FileName, header = TRUE, sep = '\t', 
		stringsAsFactors = FALSE)
	genes.class <- read.delim(FileName, header = TRUE, sep = '\t', 
		stringsAsFactors = FALSE)
\end{verbatim}
\end{framed}

Next, calculate the principal components using {\it prcomp} and plot the first two components. Then plot the first two components using {\it ggbiplot} \cite{Vu2016}:    

\begin{framed}
\begin{verbatim}
## Computing the principal components using prcomp
        genes.pca <- prcomp(gene_expression_data)
        
	# Plot the components using ggbiplot
        library(ggbiplot)
        
        g <- ggbiplot(genes.pca, obs.scale = 1, var.scale = 1,
                      groups = genes.class$groups, ellipse = TRUE,
                      circle = FALSE, var.axes = FALSE) +
          scale_color_discrete(name = '') +
          theme(legend.direction = 'horizontal', legend.position = 'top')
        
        print(g)
\end{verbatim}
\end{framed}


\noindent {\it ggbiplot} produces a plot which is shown in Figure \ref{fig:PCAFattyLiver}.
Each dot is a gene expression from a sample in each category (group) from a patient, and is coloured by its type.
The two axis are the first two principal components and the numbers represent the percentage of variance that is captured by each component. Typically, the first three component captures the most variance, whereas following components capture only a small percentage of variance.
Dots of the same type tend to cluster together which means that samples of the same type have similar expression profiles.
The distance on the dots on each axis should not be treated equally (as each component captures a different percentage of variance). Thus, the difference on the first component should be taken into more consideration.
Furthermore, the ellipse in the figure represents the normal data ellipse for each group for the details of 68\%. 

\begin{figure}[!ht]
	\includegraphics[width=\textwidth]{example2-first2components-fattyLiverData}
	\caption{The first two principal components for gene expression data in a study on Fatty Liver. The plot was generated using the R package ggbiplot \cite{Vu2016}}
	\label{fig:PCAFattyLiver}
\end{figure}




\paragraph{Extra exercise: plotting the three principal components}
\paragraph{}

Plot the three principal components for the expression data.
Figure \ref{fig:example2-PCA-3D} shows an example of such plot.

\begin{figure}[!ht]
	\includegraphics[width=\textwidth]{example2-PCA-3D}
	\caption{The first three principal components for gene expression data in a study on Fatty Liver \cite{Wruck2015}. Note: the plot was generated using MATLAB, can you plot the three components in R}
	\label{fig:example2-PCA-3D}
\end{figure}


\paragraph{Summary}
\paragraph{}

In summary, PCA is a method of revealing underling trends in large amounts of data. PCA reduces high dimensional data to just a few principal components which hopefully capture most of the variation of the data and allows inferring meaningful structure.

A new coordinate system is constructed by rotating the axes (each representing a gene). The first new coordinate, or first principal component, is the direction in which the data varies most, then the second component, and so on. PCA allows to select a few new variables which contain most of the variation of the data which can also be visualised.


Some of the benefits of using PCA benefits are:

\begin{itemize}

   \item A powerful tool to visualise high dimensional data 

   \item Shows quantified difference among observations

   \item Used to assess data quality and discover relationships between data points
   
   \item Some software to compute PCA is available in MATLAB and R (using package stats)

\end{itemize}





\section{Hierarchical Clustering}

Hierarchical Clustering is another way to visualise high dimensional data. 
It clusters observations by distance and builds a hierarchical structure.
It gives more detailed information of the differences among clusters, for example, what genes contributes the most to the differences between two clusters.


Hierarchical clustering uses a distance metric (typically Euclidean but could be correlation, Hamming distance, etc.) between each pair of genes to create a hierarchical tree-like structure of the data. Then it uses a linkage function to calculate the distance between clusters. For more details please see \cite{Clustergram}. 


Figure \ref{fig:hierarchical-clustering} shows an example of clustergram from gene expression data.
The clustergram is made of a heat map in the middle and dendograms in the left and top, with row and column labels on the right and bottom (depending on the number of genes and samples) and a scale bar. 
Each column is a sample expression profile, and each row represents a gene.
The colours suggests relative expression values, where red indicates high expression values and blue indicates low expression values. 
Ideally, samples of the same type will cluster together, e.g., all control samples will cluster together and all cases as well.

\begin{figure}[!ht]
	\includegraphics[width=\textwidth]{hierarchical-clustering}
	\caption{Example of hierarchical clustering using the MATLAB function clustergram}
	\label{fig:hierarchical-clustering}
\end{figure}


\subsection{Exercise: Hierarchical clustering for expression data}

Compute hierarchical clustering on the Fatty Liver \cite{Wruck2015}.
Since MATLAB is a commercial software, we will use a free to use wrapper for MATLAB's clustergram function. This wrapper is available from MultiPEN (\url{https://github.com/TGAC/MultiPEN/}, \cite{Rey2017}). The wrapper provided in MultiPEN reads the expression data provided as a tabular file and plots the hierarchical clustering image on screen, which is also saved as a png image. 

\subsubsection{Using MultiPEN for hierarchical clustering}

MultiPEN is shared as a MATLAB stand-alone application, which requires the installation of the MATLAB Runtime. To do so:

\begin{enumerate}

   \item Download and save MATLAB Runtime for R2015b for your operating system from: \url{http://www.mathworks.com/products/compiler/mcr/index.html}.

   \item Double click the installer and follow the instructions in the installation wizard.

\end{enumerate}


Then,  to use the MultiPEN's wrapper for hierarchical clustering, open a terminal and navigate to the folder where MultiPEN is located, then use the bash script provided as example: 

\begin{framed}
\begin{verbatim}
   ./example_hierarchical_clustering
\end{verbatim}
\end{framed}


\noindent which runs the following script:

\begin{framed}
\begin{verbatim}
   mcrpath="mcrroot/"
   OutputDirectory="ExampleOutputs/"
   ExpressionData="ExampleInputs/expressionData.txt"

   # Run MultiPEN: HierarchicalClustering
   # Syntax: MultiPEN $mcrpath HierarchicalClustering $OutputDirectory 
   $ExpressionData $Threshold $PlotTitle

   ./MultiPEN $mcrpath HierarchicalClustering $OutputDirectory 
   $ExpressionData 100 "Using example expression data"
   
\end{verbatim}
\end{framed}

\noindent where {\it mcrpath} corresponds to the location where the MATLAB Runtime is located, thus change it accordingly (see next section for the description of MultiPEN syntax). The script {\it example\_hierarchical\_clustering.sh} performs hierarchical clustering for the expression data provided in the file {\it expressionData.txt} and will save the plot as png figure in the file: {\it ExampleOutputs/hierarchical\_clustering.png}. The value 100 represents a threshold for the counts on gene expression (see syntax in below section), i.e., this function keeps the genes with counts larger than the specified threshold. 


\paragraph{Syntax}
\paragraph{}

The syntax to call the hierarchical clustering function from MultiPEN is:

\begin{framed}
   {\bf MultiPEN} {\it mcrpath} {\bf HierarchicalClustering} {\it OutputDirectory} 
   {\it ExpressionData} {\it Threshold} {\it Title}
\end{framed}


\paragraph{Description}
\paragraph{}


{\bf MultiPEN}:  This is the nombre of the software, which has to be typed to start the program.

{\it mcrpath}: MultiPEN is shared as a MATLAB stand-alone application, which requires the installation of the MATLAB Runtime. This is the path to the MATLAB run time compiler.

{\bf HierarchicalClustering}: specifies the function to run.

{\it OutputDirectory}:  Specify directory to save the output image. The default is: {\it output/MultiPEN/stats/} 

{\it ExpressionData}: The expression data is in tabular format where the rows represent the features (e.g. genes) and the colums are the samples. 

{\it Threshold}: To filter expression values. For example, for gene expression, it is common practice to discard genes with counts smaller than 100. This is an optional input argument.

{\it Title}: Specify the title to be displayed in the plot. This is an optional input argument.




\subsubsection{Exercise: Run hierarchical clustering for the Fatty Liver Data \cite{Wruck2015}}

Run the the wrapper in MultiPEN to perform hierarchical clustering in the expression data for the Fatty Liver data \cite{Wruck2015} used in previous exercise. For more information on running MultiPEN visit \url{https://github.com/TGAC/MultiPEN/} \cite{Rey2017}. Make sure the expression data is in the right format as expected from MultiPEN (tip: see format from the expression data file provided as example: MultiPEN\_v001\_Linux/ExampleInputs/expressionData.txt)




\section{Interaction Network}

\subsection{Exercise: Building a network using PSICQUIC}

Build Protein-Protein Interaction network with PSICQUIC, \url{http://www.ebi.ac.uk/Tools/webservices/psicquic/view/home.xhtml}

PSICQUIC is a tool developed by Pablo Porras at EBI and it is in principle a meta-server that runs queries of PPI from multiple primary resources. It is available as:

\begin{enumerate}
   
   \item Web-based tool http://www.ebi.ac.uk/Tools/webservices/psicquic/view/home.xhtml
   
   \item R/Bioconductor http://bioconductor.org/packages/release/bioc/html/PSICQUIC.html
   
   \item Perl or Python client script to run web queries through API

\end{enumerate}


\subsubsection{Using PSICQUIC web interface}


\begin{enumerate}
   \item Paste list of genes into the search query and select source db (see Figure \ref{fig:psicquic1}).


   \begin{figure}[!ht]
	\centering
	\includegraphics[width=\textwidth]{psicquic1}
	\caption{Query of gene using PSICQUIC web interface}
	\label{fig:psicquic1}
   \end{figure}

   \item After initial search the service informs about number of interactions detected in given resource (see Figure \ref{fig:psicquic2}).

   \begin{figure}[!ht]
	\centering
	\includegraphics[width=\textwidth]{psicquic2}
	\caption{Result of the interactions found for search term in PSICQUIC}
	\label{fig:psicquic2}
   \end{figure}
   
   \item Now we can view results, customise list of displayed columns and download results (see Figure \ref{fig:psicquic3}).
   

   \begin{figure}[!ht]
	\centering
	\includegraphics[width=\textwidth]{psicquic3}
	\caption{Customising view of results provided by PSICQUIC}
	\label{fig:psicquic3}
   \end{figure}


   \item Some of the resources e.g., Intact, Mentha, Reactome provide evidence scores that could be used for quality control purpose. Similarly as with StringDB is it advisable to use interactions with weight $>0.8$  in order to decrease number of spurious interactions. This cut-off is arbitrary and should be adjusted for specific application. 

\end{enumerate}


\subsection{Compile the network with STRINGdb}

Type the following code in an R script:

\begin{framed}
\begin{verbatim}
  # Get StringInteractome Network 
  
  #INPUT 
  # fileName - table with column 'name'
  # Uncomment next two lines and add values accordingly
  # speciesName = 'human'
  # speciesCode = 9606  #homo sapiens
  
  # networkFileName = "network.csv"   # output file name

  # Load file with list of genes  
  # Read data table
  # with at least a column "name" for the list of genes
  inputData <- read.delim( fileName, header = TRUE, 
      		sep = '\t', stringsAsFactors = FALSE)
  geneList <- inputData$name
  
  # begin compiling network
  library(STRINGdb)
  string_db <- STRINGdb$new( version="10", species = 9606, 
  		score_threshold=0, input_directory="" )
  mapped <- string_db$map( inputData,  "name", removeUnmappedRows = TRUE )
  
  #get interactions 
  inter<-string_db$get_interactions(mapped$STRING_id)
  
  #annotate source and target nodes
  from <- gsub("9606.","",inter$from)
  to <- gsub("9606.","",inter$to)
  #divide combined_score values by 1000 to have
  #scores in the range [0,1]
  network <- cbind(from,to,inter[16]/1000)  
  threshold <- 0.6   # select some relevant threshold
  subNetwork <- network[network$combined_score > threshold,] 
  
  #edit STRING_id (speciesCode.ENSPxxxxx) to remove speciesCode.
  stringID <- gsub(paste(speciesCode, ".", sep = ""), "", mapped$STRING_id)
  drops <- "STRING_id"
  mapped$STRINGID <- mapped$STRING_id
  mapped <- mapped[!(names(mapped) %in% drops)]
  
  # end compiling network
  
  #write two files: 
  #1) all network edges and 
  #2) edges above specified threshold
  cat(sprintf('\nSaving network (edges) to file: %s', fileName))
  cat('. . .')
  fileName <- paste(networkFileName, '.txt', sep = "")
  write.table(mapped, fileName, sep = '\t', col.names = TRUE, 
  		row.names = FALSE, quote = FALSE)
  cat(sprintf('Done!'))
  
  cat(sprintf('\nSaving network for threshold: 0.60 in file: %s', fileName))
  cat('. . .')
  fileName <- paste(networkFileName, 'score_0.60.txt', sep = "")
  write.table(subNetwork, fileName, sep = '\t', col.names = TRUE, 
  		row.names = FALSE, quote = FALSE)
  cat(sprintf('Done!'))
\end{verbatim}
\end{framed}


\part{Statistical Approaches with Sparsity}

\section{Feature Selection using logistic regression}

MultiPEN provides a framework for a combined analysis when different type of omics data available for a given study, specifically when data from RNA sequencing or microarray and metabolomics is available. This combined analysis aims to identify two aspects: first, the genes and metabolites that are key for the differences between two conditions, e.g., healthy vs disease, or patients under different treatments; and second, the pathway and biological processes involved in each condition. 

MultiPEN sets the problem of finding the genes and metabolites that are key to discriminate between conditions  as a problem of logistic regression, where two conditions, i.e., control and cases, are described by a set of features (i.e., genes and metabolites) and their values (i.e., expression) for two conditions. 
We apply a penalised logistic regression approach to perform feature selection which solves Equation \ref{eq:findingWeights} to find weights for the features, where $w_i$ is the weight for the ith feature.

\begin{equation}
\min_{{\bf w},v} f( {\bf w}, v ) + \lambda \Omega ({\bf w}),
\label{eq:findingWeights}
\end{equation}

\noindent where $f({\bf w}, v)$ is the expected logistic loss

\begin{equation}
f({\bf w}, v) = \frac{1}{n} \sum_{i=1}^{n}  \log( 1 + \exp (-y({\bf w}^{T}{\bf x}_i + v ) ) ),
\label{eq:ExpectedLoss}
\end{equation}

\noindent and $\Omega({\bf w}$ is a penalty function that regularises ${\bf w}$, where $\lambda \in R_+$ controls the tradeoff. GenePEN proposes a function that penalises the differences between the absolute values of weights of neighbouring features in the graph:

\begin{equation}
\Omega({\bf w}) = \sum_{i=1}^{p}  \bigg[  \sum_{j=1}^{p} A_{ij} \lvert w_i \rvert - \sum_{j=1}^{p} A_{ij} \lvert w_j \rvert  \bigg]^2 + 2 \Delta \norm{ {\bf w} }_{1}^{2}
\label{eq:penaltyFunction}
\end{equation}
 

Solving Equation \ref{eq:findingWeights} will result in finding the list of weights, {\bf w}, which provide a measure of the feature's relative importance in the learned model. The weight's absolute value can be used to rank the features, and selecting the most important could be easily achieved by setting a threshold to only consider those weights sufficiently far from zero. 


\subsection{Exercise: Features selection from expression data using MultiPEN}

%objective
For this exercise we will use MultiPEN, which provides a wrapper for to solve the Equation \ref{eq:findingWeights} for feature selection. 
MultiPEN is a software for the combined analysis of transcriptomics and metabolomics which includes a wrapper for feature selection, based on the logistic regression approach proposed by \cite{Vlassis2015}.
To perform feature selection in MultiPEN, we will use the function {\bf FeatureSelection}. 
For this example we will use the gene expression data provided with the software. 
MultiPEN is available from: \url{https://github.com/TGAC/MultiPEN/}, \cite{Rey2017}. For more details on how to use the software see \url{https://github.com/TGAC/MultiPEN/blob/master/MultiPEN_executable/MultiPEN_v001_documentation/user-manual.md}

%We will use the example gene expression data on the Fatty Liver study \cite{Wruck2015} and apply a logistic regression for feature selection as proposed by \cite{Vlassis2015}. We will use the feature selection function (called: FeatureSelection) provided by the software MultiPEN, which is available from: \url{https://github.com/TGAC/MultiPEN/}, \cite{Rey2017}. 

\subsubsection{Using the Application}


MultiPEN is shared as a MATLAB stand-alone application, which requires the installation of the MATLAB Runtime. To do so:

\begin{enumerate}

   \item Download and save MATLAB Runtime for R2015b for your operating system from: \url{http://www.mathworks.com/products/compiler/mcr/index.html}.

   \item Double click the installer and follow the instructions in the installation wizard.

\end{enumerate}


\subsubsection{Running MultiPEN and the function FeatureSelection}


For this session, we included for you the application folder, called {\bf MultiPEN\_v001\_Linux}. The content of this folder is described below:

\begin{outline}

   \1 {\bf MultiPEN} is the application. To launch it, open a terminal, navigate to the application location (e.g., if the application is saved in Applications/, then navigate to: Applications/MultiPEN\_v001\_Linux). Type:
   
   {\bf MultiPEN}    {\it mcrpath}   [...options] 
   
   for more details on the syntax see next section.

   \1 {\bf ExampleInputs}. Expression data provided to test the application. It includes:
      \2 expressionData.txt
      \2 groups.txt
      \2 interactionMatrix.txt
      \2 MultiPEN-Rankings\_lambda0.0001-onlyGenes.txt
      \2 sampleClass.txt
   
   \1 {\bf ExampleOutputs}. Examples of the output files generated with MultiPEN for feature selection, cross validation and enrichment analysis using the toy example data.
      \2 Cross Validation
         \3 MultiPEN-performance\_feature-selection\_lambda0.0001.txt
         
      \2 Feature Selection
      
      \2 Enrichment Analysis
         \3 enrichment-GO\_BP.pdf
         \3 enrichment-GO\_CC.pdf
         \3 enrichment-GO\_MF.pdf
         \3 enrichment-GO.txt
         
      \2 MultiPEN-feature-selection\_config.txt
         \3 MultiPEN-Rankings\_lambda0.0001\_higher-in-cases.txt
         \3 MultiPEN-Rankings\_lambda0.0001\_higher-in-control.txt
         \3 MultiPEN-Rankings\_lambda0.0001.txt
         \3 MultiPEN-vts\_lambda0.0001.txt
   
   \1 Bash scripts with examples to run the application MultiPEN.
   
      \2 example\_cross\_validation.sh
      \2 example\_feature\_selection.sh
      \2 example\_hierarchical\_clustering.sh
      \2 example\_pca.sh

\end{outline}



\paragraph{Syntax}
\paragraph{}


To run MultiPEN open a terminal, navigate to the folder (MultiPEN\_v001\_Linux) where the software is located. Then use the following syntax to run feature selection. The words in bold indicate the name of the program (MultiPEN) and the specific function that will be used (e.g. FeatureSelection).

\begin{framed}
%\begin{verbatim}
{\bf MultiPEN} {\it mcrpath }  {\bf FeatureSelection } {\it OutputDirectory} 
{\it ExpressionData } {\it Interactions } {\it SampleClass } {\it lambda } 
{\it DecisionThreshold } {\it NumIterations}
%\end{verbatim}
\end{framed}


\paragraph{Description}
\paragraph{}


{\bf MultiPEN}:  This is the nombre of the software, which has to be typed to start the program.

{\it mcrpath}: MultiPEN is shared as a MATLAB stand-alone application, which requires the installation of the MATLAB Runtime. This is the path to the MATLAB run time compiler.

{\bf FeatureSelection}: specifies the function to run.

{\it OutputDirectory}:  Specify directory to save the output image. The default is: {\it output/MultiPEN/stats/} 

{\it ExpressionData}: The expression data is in tabular format where the rows represent the features (e.g. genes) and the colums are the samples.

{\it Interactions}: The interaction matrix where the i{\it th} interaction (row) is represented as: [source target score] where {\it source} and {\it target} are names (symbolID for genes and CHEBI IDs for metabolites) of the connected nodes and {\it score} is a number in the range [0,1] representing the interaction confidence (where 1 corresponds to the maximum level of confidence).

{\it SampleClass}: For each sample (one sample per row) specify if control (0) or case (1).

{\it lambda}: This is the lambda parameter that optimises the logistic regression problem for your specific data. Different lambdas can be tested using cross validation, then selecting the value that provides better results (in terms of the size of the largest connected component, accuracy or area under the curve).  

{\it DecisionThreshold}: The decision threshold is set to 0.5 by default. However, if want to test another value specify it here.

{\it NumIterations}: Maximum number of iterations for the optimisation solver. Default value is 100.


\subsubsection{Run MultiPEN and FeatureSelection with the example data}


To run MultiPEN open a terminal and navigate to the folder (MultiPEN\_v001\_Linux) where the software is located. Then, set up the input parameters by typing:

\begin{framed}
\begin{verbatim}
   mcrpath="mcrroot/" 
   OutputDirectory="ExampleOutputs/"
   ExpressionData="ExampleInputs/expressionData.txt"
   Interactions="ExampleInputs/interactionMatrix.txt"
   SampleClass="ExampleInputs/sampleClass.txt"
   lambda=0.001
\end{verbatim}
\end{framed}


Next, run the function FeatureSelection from MultiPEN:

\begin{framed}
\begin{verbatim}
   MultiPEN $mcrpath FeatureSelection $OutputDirectory
   $ExpressionData $Interactions $SampleClass $lambda
\end{verbatim}
\end{framed}


The above will produce the following output files:

\subsubsection{Feature Selection Output Files}


Feature selection produces seven output files: 

\paragraph{MultiPEN-Rankings\_lambdaX.txt:} Ranking of features for the corresponding lambda X. 
This file contains the following columns and (n+5) rows (where n is the number of samples):

\paragraph{}

\begin{tabulary}{\linewidth}{c c L c}
Column & Column Name & Description & Example (row 4 in Figure \ref{fig:example_output_rankings}) \\
\hline
1 & name & Feature name & PPOX \\
2 & weight & Weight (in the range [-1,1]) & 0.00290391 \\
3 & ranking & Ranking according to the absolute weight, where ranking 1 corresponds to the most significant feature for the model & 3 \\
4 & foldChange & Fold change to determine the expression change from control to cases & 1.1735 \\
5 & higherIn & The average expression is higher in case or control & case \\
6 & sample\_1 & First sample & case1 \\
. . .  & . . .  & . . .  & . . . \\
n + 5 & sample\_n & Last sample & control7 \\
\end{tabulary}

\paragraph{MultiPEN-Rankings\_lambdaX\_genes-higher-in-cases.txt:} Ranking of features which includes only features with higher expression in cases samples.

\paragraph{MultiPEN-Rankings\_lambdaX\_genes-higher-in-control.txt:} Ranking of features which includes only features with higher expression in control samples.

\paragraph{MultiPEN-vts\_lambdaX.txt:} Intercept term (logistic regression model).

\paragraph{MultiPEN-performance\_feature-selection\_lambdaX.txt:} file that contains statistics on the performance of feature selection: largest connected component (LCC), area under the curve (auc), accuracy, true positives (TP), true negatives (TN), false positives (FP) and false negatives (FN).

\paragraph{MultiPEN-feature-selection\_config.txt:} Contains the information of the parameters used: lambda, number of iterations and decision threshold.



\begin{figure}[!ht]
	\centering
	\includegraphics[width=\textwidth]{example_output_rankings}
	\caption{Example of output file for feature selection}
	\label{fig:example_output_rankings}
\end{figure}



\subsection{Cross Validation}

It is a technique to determine whether a model can be generalised to other similar databases (it measure the accuracy of a model).

\paragraph{Approach:}


\begin{itemize}

   \item Divide the dataset into training and test datasets
   
   \item Fit the model to the training set
   
   \item Use test set to evaluate goodness of fit
   
\end{itemize}


Theory

\begin{itemize}

   \item Signal is correlated across tests and training sets
   
   \item Noise is uncorrelated across tests and training sets
   
\end{itemize}




We will use the wrapper provided in MultiPEN to perform cross validation. 

\paragraph{Syntax}
\paragraph{}


\begin{framed}

   {\bf MultiPEN} {\it mcrpath} {\bf CrossValidation} {\it OutputDirectory} {\it ExpressionData} {\it Interactions} 
   {\it SampleClass} {\it lambdas} {\it Folds} {\it  NumIterations}

\end{framed}


\paragraph{Description}
\paragraph{}


{\bf MultiPEN}:  This is the nombre of the software, which has to be typed to start the program.

{\it mcrpath}: MultiPEN is shared as a MATLAB stand-alone application, which requires the installation of the MATLAB Runtime. This is the path to the MATLAB run time compiler.

{\bf CrossValidation}: specifies the function to run.

{\it OutputDirectory}:  Specify directory to save the output image. The default is: {\it output/MultiPEN/stats/} 

{\it ExpressionData}: The expression data is in tabular format where the rows represent the features (e.g. genes) and the colums are the samples.

{\it Interactions}: The interaction matrix where the i{\it th} interaction (row) is represented as: [source target score] where {\it source} and {\it target} are names (symbolID for genes and CHEBI IDs for metabolites) of the connected nodes and {\it score} is a number in the range [0,1] representing the interaction confidence (where 1 corresponds to the maximum level of confidence).

{\it SampleClass}: For each sample (one sample per row) specify if control (0) or case (1).

{\it lambdas}: Set of lambdas to test for cross validation. If you are wanting to test more than one lambda, specify the lambdas by using the notation (include the quotation mark symbols): ``[lambda1 lambda2 . . . lambdaN]''. For example, if we want to try two lambdas, namely 0.02 and 0.2, we would specify it with: ``[0.02 0.2]''.

{\it Folds}: Specify the number of partitions for cross validation.

{\it NumIterations}: Maximum number of iterations for the optimisation solver. Default value is 100.





\subsubsection{Running MultiPEN for cross validation}
\paragraph{}


In the command line, navigate to the folder where the binary for MultiPEN is located, i.e., MultiPEN\_v001\_Linux/. Then create variables for the paths to stand-alone application, output directory and input files by typing:


\begin{framed}
\begin{verbatim}
   OutputDirectory="ExampleOutputs/"
   mcrpath="mcroot/"
   ExpressionData="ExampleInputs/X.txt"
   Interactions="ExampleInputs/E.txt"
   SampleClass="ExampleInputs/Y.txt"
   Folds=3
   NumIter=3000
\end{verbatim}
\end{framed}


Note that in this example we are using the example files provided with the application. All the files used for the example are located in the folder: ExampleInputs/.


Next, test Cross Validation for lambdas: "[0.000001 0.00001 0.0001 0.001 0.01 0.1 1 10]". Type the following command:


\begin{framed}
\begin{verbatim}
   MultiPEN $mcrpath CrossValidation $OutputDirectory 
   $ExpressionData $Interactions $SampleClass 
   "[0.000001 0.00001 0.0001 0.001 0.01 0.1 1 10]" $Folds $NumIter
\end{verbatim}
\end{framed}




\subsubsection{Cross Validation Output Files}


Cross Validation produces one output file:

{\bf cross-validation\_stats.txt:} Statistics for tests which include, for each lambda, the size of the largest connected component (LCC), the standard deviation of the largest connected component (std\_LCC), the number of selected features (selected, i.e., features which weights are different to zero), area under the curve (AUC), and the standard deviation of the area under the curve (std\_AUC).




\bibliography{references}
\bibliographystyle{plain}

\end{document}  